%% Common header for WG21 proposals ? mainly taken from C++ standard draft source
%%

%%--------------------------------------------------
%% basics
%\documentclass[a4paper,11pt,oneside,openany,final,article]{memoir}
\documentclass[a4paper,11pt,oneside,openany,final,article]{memoir}

\usepackage[american]
           {babel}        % needed for iso dates
\usepackage[iso,american]
           {isodate}      % use iso format for dates
\usepackage[final]
           {listings}     % code listings
\usepackage{longtable}    % auto-breaking tables
\usepackage{ltcaption}    % fix captions for long tables
\usepackage{relsize}      % provide relative font size changes
\usepackage{textcomp}     % provide \text{l,r}angle
\usepackage{underscore}   % remove special status of '_' in ordinary text
\usepackage{parskip}      % handle non-indented paragraphs "properly"
\usepackage{array}        % new column definitions for tables
\usepackage[normalem]{ulem}
\usepackage{enumitem}
\usepackage{color}        % define colors for strikeouts and underlines
\usepackage{xcolor}    % needed for blue links

\usepackage{amsmath}      % additional math symbols
\usepackage{mathrsfs}     % mathscr font
\usepackage[final]{microtype}
\usepackage{multicol}
\usepackage{lmodern}
\usepackage[T1]{fontenc}
\usepackage[pdftex, final]{graphicx}
\usepackage[pdftex,
            bookmarks=true,
            bookmarksnumbered=true,
            pdfpagelabels=true,
            pdfpagemode=UseOutlines,
            pdfstartview=FitH,
            linktocpage=true,
            colorlinks=true,
            plainpages=false,
            allcolors={blue}, 
            allbordercolors={white}
           ]{hyperref}
\usepackage{memhfixc}     % fix interactions between hyperref and memoir
\usepackage{url}  % urls in ref.bib
\usepackage{tabularx}  % don't use the C++ standard's fancy tables, they come with captions!

\pdfminorversion=5
\pdfcompresslevel=9
\pdfobjcompresslevel=2

\renewcommand\RSsmallest{5.5pt}  % smallest font size for relsize

%%--------------------------------------------------
%%--------------------------------------------------
%% Layout -- set overall page appearance

%%--------------------------------------------------
%%  set page size, type block size, type block position

\setlrmarginsandblock{2.245cm}{2.245cm}{*}
\setulmarginsandblock{2.5cm}{2.5cm}{*}

%%--------------------------------------------------
%%  set header and footer positions and sizes

\setheadfoot{\onelineskip}{2\onelineskip}
\setheaderspaces{*}{2\onelineskip}{*}

%%--------------------------------------------------
%%  make miscellaneous adjustments, then finish the layout
\setmarginnotes{7pt}{7pt}{0pt}
\checkandfixthelayout

%%--------------------------------------------------
%% If there is insufficient stretchable vertical space on a page,
%% TeX will not properly consider penalties for a good page break,
%% even if \raggedbottom (default) is in effect.
\addtolength{\topskip}{0pt plus 20pt}

%%--------------------------------------------------
%% Paragraph and bullet numbering

\newcounter{Paras}
\counterwithout{section}{chapter}
\setcounter{secnumdepth}{3}

\newcounter{Bullets1}[Paras]
\newcounter{Bullets2}[Bullets1]
\newcounter{Bullets3}[Bullets2]
\newcounter{Bullets4}[Bullets3]

\makeatletter
\newcommand{\parabullnum}[2]{%
\stepcounter{#1}%
\noindent\makebox[0pt][l]{\makebox[#2][r]{%
\scriptsize\raisebox{.7ex}%
{%
\ifnum \value{Paras}>0
\ifnum \value{Bullets1}>0 (\fi%
                          \arabic{Paras}%
\ifnum \value{Bullets1}>0 .\arabic{Bullets1}%
\ifnum \value{Bullets2}>0 .\arabic{Bullets2}%
\ifnum \value{Bullets3}>0 .\arabic{Bullets3}%
\fi\fi\fi%
\ifnum \value{Bullets1}>0 )\fi%
\fi%
}%
\hspace{\@totalleftmargin}\quad%
}}}
\makeatother

\def\pnum{\parabullnum{Paras}{0pt}}

%%--------------------------------------------------
%%--------------------------------------------------
%% Styles
%!TEX root = std.tex
%% styles.tex -- set styles for:
%     chapters
%     pages
%     footnotes

%%--------------------------------------------------
%%  create chapter style

\makechapterstyle{cppstd}{%
  \renewcommand{\beforechapskip}{\onelineskip}
  \renewcommand{\afterchapskip}{\onelineskip}
  \renewcommand{\chapternamenum}{}
  \renewcommand{\chapnamefont}{\chaptitlefont}
  \renewcommand{\chapnumfont}{\chaptitlefont}
  \renewcommand{\printchapternum}{\chapnumfont\thechapter\quad}
  \renewcommand{\afterchapternum}{}
}

%%--------------------------------------------------
%%  create page styles




%%--------------------------------------------------
% set style for main text
\setlength{\parindent}{0pt}
\setlength{\parskip}{1ex}

%%--------------------------------------------------
%% change list item markers to number and em-dash

\renewcommand{\labelitemi}{---\parabullnum{Bullets1}{\labelsep}}
\renewcommand{\labelitemii}{---\parabullnum{Bullets2}{\labelsep}}
\renewcommand{\labelitemiii}{---\parabullnum{Bullets3}{\labelsep}}
\renewcommand{\labelitemiv}{---\parabullnum{Bullets4}{\labelsep}}



%%--------------------------------------------------
%% override some functions from the listings package to avoid bad page breaks
%% (copied verbatim from listings.sty version 1.6 except where commented)
\makeatletter


\def\lst@Init#1{%
    \begingroup
    \ifx\lst@float\relax\else
        \edef\@tempa{\noexpand\lst@beginfloat{lstlisting}[\lst@float]}%
        \expandafter\@tempa
    \fi
    \ifx\lst@multicols\@empty\else
        \edef\lst@next{\noexpand\multicols{\lst@multicols}}
        \expandafter\lst@next
    \fi
    \ifhmode\ifinner \lst@boxtrue \fi\fi
    \lst@ifbox
        \lsthk@BoxUnsafe
        \hbox to\z@\bgroup
             $\if t\lst@boxpos \vtop
        \else \if b\lst@boxpos \vbox
        \else \vcenter \fi\fi
        \bgroup \par\noindent
    \else
        \lst@ifdisplaystyle
            \lst@EveryDisplay
            % make penalty configurable
            \par\lst@beginpenalty
            \vspace\lst@aboveskip
        \fi
    \fi
    \normalbaselines
    \abovecaptionskip\lst@abovecaption\relax
    \belowcaptionskip\lst@belowcaption\relax
    \lst@MakeCaption t%
    \lsthk@PreInit \lsthk@Init
    \lst@ifdisplaystyle
        \global\let\lst@ltxlabel\@empty
        \if@inlabel
            \lst@ifresetmargins
                \leavevmode
            \else
                \xdef\lst@ltxlabel{\the\everypar}%
                \lst@AddTo\lst@ltxlabel{%
                    \global\let\lst@ltxlabel\@empty
                    \everypar{\lsthk@EveryLine\lsthk@EveryPar}}%
            \fi
        \fi
        % A section heading might have set \everypar to apply a \clubpenalty
        % to the following paragraph, changing \everypar in the process.
        % Unconditionally overriding \everypar is a bad idea.
        % \everypar\expandafter{\lst@ltxlabel
        %                      \lsthk@EveryLine\lsthk@EveryPar}%
    \else
        \everypar{}\let\lst@NewLine\@empty
    \fi
    \lsthk@InitVars \lsthk@InitVarsBOL
    \lst@Let{13}\lst@MProcessListing
    \let\lst@Backslash#1%
    \lst@EnterMode{\lst@Pmode}{\lst@SelectCharTable}%
    \lst@InitFinalize}

\def\lst@DeInit{%
    \lst@XPrintToken \lst@EOLUpdate
    \global\advance\lst@newlines\m@ne
    \lst@ifshowlines
        \lst@DoNewLines
    \else
        \setbox\@tempboxa\vbox{\lst@DoNewLines}%
    \fi
    \lst@ifdisplaystyle \par\removelastskip \fi
    \lsthk@ExitVars\everypar{}\lsthk@DeInit\normalbaselines\normalcolor
    \lst@MakeCaption b%
    \lst@ifbox
        \egroup $\hss \egroup
        \vrule\@width\lst@maxwidth\@height\z@\@depth\z@
    \else
        \lst@ifdisplaystyle
            % make penalty configurable
            \par\lst@endpenalty
            \vspace\lst@belowskip
        \fi
    \fi
    \ifx\lst@multicols\@empty\else
        \def\lst@next{\global\let\@checkend\@gobble
                      \endmulticols
                      \global\let\@checkend\lst@@checkend}
        \expandafter\lst@next
    \fi
    \ifx\lst@float\relax\else
        \expandafter\lst@endfloat
    \fi
    \endgroup}


\def\lst@NewLine{%
    \ifx\lst@OutputBox\@gobble\else
        \par
        % add configurable penalties
        \lst@ifeolsemicolon
          \lst@semicolonpenalty
          \lst@eolsemicolonfalse
        \else
          \lst@domidpenalty
        \fi
        % Manually apply EveryLine and EveryPar; do not depend on \everypar
        \noindent \hbox{}\lsthk@EveryLine%
        % \lsthk@EveryPar uses \refstepcounter which balloons the PDF
    \fi
    \global\advance\lst@newlines\m@ne
    \lst@newlinetrue}

% new macro for empty lines, avoiding an \hbox that cannot be discarded
\def\lst@DoEmptyLine{%
  \ifvmode\else\par\fi\lst@emptylinepenalty
  \vskip\parskip
  \vskip\baselineskip
  % \lsthk@EveryLine has \lst@parshape, i.e. \parshape, which causes an \hbox
  % \lsthk@EveryPar increments line counters; \refstepcounter balloons the PDF
  \global\advance\lst@newlines\m@ne
  \lst@newlinetrue}

\def\lst@DoNewLines{
    \@whilenum\lst@newlines>\lst@maxempty \do
        {\lst@ifpreservenumber
            \lsthk@OnEmptyLine
            \global\advance\c@lstnumber\lst@advancelstnum
         \fi
         \global\advance\lst@newlines\m@ne}%
    \@whilenum \lst@newlines>\@ne \do
        % special-case empty printing of lines
        {\lsthk@OnEmptyLine\lst@DoEmptyLine}%
    \ifnum\lst@newlines>\z@ \lst@NewLine \fi}

% add keys for configuring before/end vertical penalties
\lst@Key{beginpenalty}\relax{\def\lst@beginpenalty{\penalty #1}}
\let\lst@beginpenalty\@empty
\lst@Key{midpenalty}\relax{\def\lst@midpenalty{\penalty #1}}
\let\lst@midpenalty\@empty
\lst@Key{endpenalty}\relax{\def\lst@endpenalty{\penalty #1}}
\let\lst@endpenalty\@empty
\lst@Key{emptylinepenalty}\relax{\def\lst@emptylinepenalty{\penalty #1}}
\let\lst@emptylinepenalty\@empty
\lst@Key{semicolonpenalty}\relax{\def\lst@semicolonpenalty{\penalty #1}}
\let\lst@semicolonpenalty\@empty

\lst@AddToHook{InitVars}{\let\lst@domidpenalty\@empty}
\lst@AddToHook{InitVarsEOL}{\let\lst@domidpenalty\lst@midpenalty}

% handle semicolons and closing braces (could be in \lstdefinelanguage as well)
\def\lst@eolsemicolontrue{\global\let\lst@ifeolsemicolon\iftrue}
\def\lst@eolsemicolonfalse{\global\let\lst@ifeolsemicolon\iffalse}
\lst@AddToHook{InitVars}{
  \global\let\lst@eolsemicolonpending\@empty
  \lst@eolsemicolonfalse
}
% If we found a semicolon or closing brace while parsing the current line,
% inform the subsequent \lst@NewLine about it for penalties.
\lst@AddToHook{InitVarsEOL}{%
  \ifx\lst@eolsemicolonpending\relax
    \lst@eolsemicolontrue
    \global\let\lst@eolsemicolonpending\@empty
  \fi%
}
\lst@AddToHook{SelectCharTable}{%
  % In theory, we should only detect trailing semicolons or braces,
  % but that would require un-doing the marking for any other character.
  % The next best thing is to undo the marking for closing parentheses,
  % because loops or if statements are the only places where we will
  % reasonably have a semicolon in the middle of a line, and those all
  % end with a closing parenthesis.
  \lst@DefSaveDef{41}\lstsaved@closeparen{%    handle closing parenthesis
    \lstsaved@closeparen
    \ifnum\lst@mode=\lst@Pmode    % regular processing mode (not a comment)
      \global\let\lst@eolsemicolonpending\@empty  % undo semicolon setting
    \fi%
  }%
  \lst@DefSaveDef{59}\lstsaved@semicolon{%     handle semicolon
    \lstsaved@semicolon
    \ifnum\lst@mode=\lst@Pmode    % regular processing mode (not a comment)
      \global\let\lst@eolsemicolonpending\relax
    \fi%
  }%
  \lst@DefSaveDef{125}\lstsaved@closebrace{%   handle closing brace
    \lst@eolsemicolonfalse        % do not break before a closing brace
    \lstsaved@closebrace          % might invoke \lst@NewLine
    \ifnum\lst@mode=\lst@Pmode    % regular processing mode (not a comment)
      \global\let\lst@eolsemicolonpending\relax
    \fi%
  }%
}

\makeatother


%%--------------------------------------------------
%%--------------------------------------------------
%% Macros
%!TEX root = std.tex
% Definitions and redefinitions of special commands

%%--------------------------------------------------
%% Difference markups
\definecolor{addclr}{rgb}{0,0.4,0.05}
\definecolor{remclr}{rgb}{1,0,0}
\definecolor{noteclr}{rgb}{0,0,1}

\renewcommand{\added}[1]{\textcolor{addclr}{\uline{#1}}}
\newcommand{\removed}[1]{\textcolor{remclr}{\sout{#1}}}
\renewcommand{\changed}[2]{\removed{#1}\added{#2}}

\newcommand{\nbc}[1]{[#1]\ }
\newcommand{\addednb}[2]{\added{\nbc{#1}#2}}
\newcommand{\removednb}[2]{\removed{\nbc{#1}#2}}
\newcommand{\changednb}[3]{\removednb{#1}{#2}\added{#3}}
\newcommand{\remitem}[1]{\item\removed{#1}}

\newcommand{\ednote}[1]{\textcolor{noteclr}{[Editor's note: #1] }}
% \newcommand{\ednote}[1]{}

\newenvironment{addedblock}
{
\color{addclr}
}
{
\color{black}
}
\newenvironment{removedblock}
{
\color{remclr}
}
{
\color{black}
}

%%--------------------------------------------------
% General code style
\newcommand{\CodeStyle}{\ttfamily}
\newcommand{\CodeStylex}[1]{\texttt{#1}}
\newcommand{\CodeStylexul}[1]{\underline{\texttt{#1}}}

% Code and definitions embedded in text.
\newcommand{\tcode}[1]{\CodeStylex{#1}}
\newcommand{\tcodeul}[1]{\CodeStylexul{#1}}
\newcommand{\techterm}[1]{\textit{#1}}
\newcommand{\defnx}[2]{\indexdefn{#2}\textit{#1}}
\newcommand{\defn}[1]{\defnx{#1}{#1}}
\newcommand{\term}[1]{\textit{#1}}
\newcommand{\grammarterm}[1]{\textit{#1}}
\newcommand{\grammartermnc}[1]{\textit{#1}\nocorr}
\newcommand{\placeholder}[1]{\textit{#1}}
\newcommand{\placeholdernc}[1]{\textit{#1\nocorr}}

%%--------------------------------------------------
%% allow line break if needed for justification
\newcommand{\brk}{\discretionary{}{}{}}

%%--------------------------------------------------
%% Macros for funky text
\newcommand{\Cpp}{\texorpdfstring{C\kern-0.05em\protect\raisebox{.35ex}{\textsmaller[2]{+\kern-0.05em+}}}{C++}}
\newcommand{\CppIII}{\Cpp{} 2003}
\newcommand{\CppXI}{\Cpp{} 2011}
\newcommand{\CppXIV}{\Cpp{} 2014}
\newcommand{\CppXVII}{\Cpp{} 2017}
\newcommand{\opt}[1]{\ifthenelse{\equal{#1}{}}
    {\PackageError{main}{argument must not be empty}{}}
    {#1\ensuremath{_\mathit{opt}}}}
\newcommand{\dcr}{-{-}}
\newcommand{\bigoh}[1]{\ensuremath{\mathscr{O}(#1)}}

% Make all tildes a little larger to avoid visual similarity with hyphens.
\renewcommand{\~}{\textasciitilde}
\let\OldTextAsciiTilde\textasciitilde
\renewcommand{\textasciitilde}{\protect\raisebox{-0.17ex}{\larger\OldTextAsciiTilde}}
\newcommand{\caret}{\char`\^}

%%--------------------------------------------------
%% States and operators
\newcommand{\state}[2]{\tcode{#1}\ensuremath{_{#2}}}
\newcommand{\bitand}{\ensuremath{\, \mathsf{bitand} \,}}
\newcommand{\bitor}{\ensuremath{\, \mathsf{bitor} \,}}
\newcommand{\xor}{\ensuremath{\, \mathsf{xor} \,}}
\newcommand{\rightshift}{\ensuremath{\, \mathsf{rshift} \,}}
\newcommand{\leftshift}[1]{\ensuremath{\, \mathsf{lshift}_#1 \,}}

%% Notes and examples
\newcommand{\noteintro}[1]{[\,\textit{#1:}\space}
\newcommand{\noteoutro}[1]{\textit{\,---\,end #1}\,]}
\newenvironment{note}[1][Note]{\noteintro{#1}}{\noteoutro{note}\space}
\newenvironment{example}[1][Example]{\noteintro{#1}}{\noteoutro{example}\space}

%% Library function descriptions
\newcommand{\Fundescx}[1]{\textit{#1}}
\newcommand{\Fundesc}[1]{\Fundescx{#1:}\space}
\newcommand{\required}{\Fundesc{Required behavior}}
\newcommand{\requires}{\Fundesc{Requires}}
\newcommand{\effects}{\Fundesc{Effects}}
\newcommand{\postconditions}{\Fundesc{Postconditions}}
\newcommand{\returns}{\Fundesc{Returns}}
\newcommand{\throws}{\Fundesc{Throws}}
\newcommand{\default}{\Fundesc{Default behavior}}
\newcommand{\complexity}{\Fundesc{Complexity}}
\newcommand{\remarks}{\Fundesc{Remarks}}
\newcommand{\errors}{\Fundesc{Error conditions}}
\newcommand{\sync}{\Fundesc{Synchronization}}
\newcommand{\implimits}{\Fundesc{Implementation limits}}
\newcommand{\replaceable}{\Fundesc{Replaceable}}
\newcommand{\returntype}{\Fundesc{Return type}}
\newcommand{\cvalue}{\Fundesc{Value}}
\newcommand{\ctype}{\Fundesc{Type}}
\newcommand{\ctypes}{\Fundesc{Types}}
\newcommand{\dtype}{\Fundesc{Default type}}
\newcommand{\ctemplate}{\Fundesc{Class template}}
\newcommand{\templalias}{\Fundesc{Alias template}}

%% Cross reference
\newcommand{\xref}{\textsc{See also:}\space}

%% Inline parenthesized reference
\newcommand{\iref}[1]{\nolinebreak[3] (\ref{#1})}

%% NTBS, etc.
\newcommand{\NTS}[1]{\textsc{#1}}
\newcommand{\ntbs}{\NTS{ntbs}}
\newcommand{\ntmbs}{\NTS{ntmbs}}
% The following are currently unused:
% \newcommand{\ntwcs}{\NTS{ntwcs}}
% \newcommand{\ntcxvis}{\NTS{ntc16s}}
% \newcommand{\ntcxxxiis}{\NTS{ntc32s}}

%% Code annotations
\newcommand{\EXPO}[1]{\textit{#1}}
\newcommand{\expos}{\EXPO{exposition only}}
\newcommand{\impdef}{\EXPO{implementation-defined}}
\newcommand{\impdefnc}{\EXPO{implementation-defined\nocorr}}
\newcommand{\impdefx}[1]{\indeximpldef{#1}\EXPO{implementation-defined}}
\newcommand{\notdef}{\EXPO{not defined}}

\newcommand{\UNSP}[1]{\textit{\texttt{#1}}}
\newcommand{\UNSPnc}[1]{\textit{\texttt{#1}\nocorr}}
\newcommand{\unspec}{\UNSP{unspecified}}
\newcommand{\unspecnc}{\UNSPnc{unspecified}}
\newcommand{\unspecbool}{\UNSP{unspecified-bool-type}}
\newcommand{\seebelow}{\UNSP{see below}}
\newcommand{\seebelownc}{\UNSPnc{see below}}
\newcommand{\unspecuniqtype}{\UNSP{unspecified unique type}}
\newcommand{\unspecalloctype}{\UNSP{unspecified allocator type}}

\newcommand{\EXPLICIT}{\textit{\texttt{EXPLICIT}\nocorr}}

%% Manual insertion of italic corrections, for aligning in the presence
%% of the above annotations.
\newlength{\itcorrwidth}
\newlength{\itletterwidth}
\newcommand{\itcorr}[1][]{%
 \settowidth{\itcorrwidth}{\textit{x\/}}%
 \settowidth{\itletterwidth}{\textit{x\nocorr}}%
 \addtolength{\itcorrwidth}{-1\itletterwidth}%
 \makebox[#1\itcorrwidth]{}%
}

%% Double underscore
\newcommand{\ungap}{\kern.5pt}
\newcommand{\unun}{\textunderscore\ungap\textunderscore}
\newcommand{\xname}[1]{\tcode{\unun\ungap#1}}
\newcommand{\mname}[1]{\tcode{\unun\ungap#1\ungap\unun}}

%% An elided code fragment, /* ... */, that is formatted as code.
%% (By default, listings typeset comments as body text.)
%% Produces 9 output characters.
\newcommand{\commentellip}{\tcode{/* ...\ */}}

%% Ranges
\newcommand{\Range}[4]{\tcode{#1#3,\penalty2000{} #4#2}}
\newcommand{\crange}[2]{\Range{[}{]}{#1}{#2}}
\newcommand{\brange}[2]{\Range{(}{]}{#1}{#2}}
\newcommand{\orange}[2]{\Range{(}{)}{#1}{#2}}
\newcommand{\range}[2]{\Range{[}{)}{#1}{#2}}

%% Change descriptions
\newcommand{\diffdef}[1]{\hfill\break\textbf{#1:}\space}
\newcommand{\diffref}[1]{\pnum\textbf{Affected subclause:} \ref{#1}}
\newcommand{\change}{\diffdef{Change}}
\newcommand{\rationale}{\diffdef{Rationale}}
\newcommand{\effect}{\diffdef{Effect on original feature}}
\newcommand{\difficulty}{\diffdef{Difficulty of converting}}
\newcommand{\howwide}{\diffdef{How widely used}}

%% Miscellaneous
\newcommand{\uniquens}{\placeholdernc{unique}}
\newcommand{\stage}[1]{\item[Stage #1:]}
\newcommand{\doccite}[1]{\textit{#1}}
\newcommand{\cvqual}[1]{\textit{#1}}
\newcommand{\cv}{\cvqual{cv}}
\newcommand{\numconst}[1]{\textsl{#1}}
\newcommand{\logop}[1]{{\footnotesize #1}}

%%--------------------------------------------------
%% Environments for code listings.

% We use the 'listings' package, with some small customizations.  The
% most interesting customization: all TeX commands are available
% within comments.  Comments are set in italics, keywords and strings
% don't get special treatment.

\lstset{language=C++,
        basicstyle=\small\CodeStyle,
        keywordstyle=,
        stringstyle=,
        xleftmargin=1em,
        showstringspaces=false,
        commentstyle=\itshape\rmfamily,
        columns=fullflexible,
        keepspaces=true,
        texcl=true}


% Our usual abbreviation for 'listings'.  Comments are in
% italics.  Arbitrary TeX commands can be used if they're
% surrounded by @ signs.
\newcommand{\CodeBlockSetup}{
 \lstset{escapechar=@, aboveskip=\parskip, belowskip=0pt,
         midpenalty=500, endpenalty=-50,
         emptylinepenalty=-250, semicolonpenalty=0,rulecolor=\color{black}}
 \renewcommand{\tcode}[1]{\textup{\CodeStylex{##1}}}
 \newcommand{\textbfx}[1]{\textup{\CodeStylex{\color{blue}\textbf{##1}}}}
 \renewcommand{\techterm}[1]{\textit{\CodeStylex{##1}}}
 \renewcommand{\term}[1]{\textit{##1}}
 \renewcommand{\grammarterm}[1]{\textit{##1}}
}

\lstnewenvironment{codeblock}{\CodeBlockSetup}{}



\newcommand{\CodeBlockAddSetup}{
 \lstset{basicstyle=\small\CodeStyle\color{addclr},escapechar=@, aboveskip=\parskip, belowskip=0pt,
         midpenalty=500, endpenalty=-50,
         emptylinepenalty=-250, semicolonpenalty=0}
 \renewcommand{\tcode}[1]{\textup{\CodeStylex{##1}}}
 \renewcommand{\techterm}[1]{\textit{\CodeStylex{##1}}}
 \renewcommand{\term}[1]{\textit{##1}}
 \renewcommand{\grammarterm}[1]{\textit{##1}}
}

\lstnewenvironment{codeblockAdd}{\CodeBlockAddSetup}{}




% Our usual abbreviation for 'listings'.  Comments are in
% italics.  Arbitrary TeX commands can be used if they're
% surrounded by @ signs.
\newcommand{\CodeBlockSetupSmall}{
 \tiny
 \lstset{escapechar=@, aboveskip=\parskip, belowskip=0pt,
         midpenalty=500, endpenalty=-50, basicstyle=\small,
         emptylinepenalty=-250, semicolonpenalty=0}
 \renewcommand{\tcode}[1]{\textup{\CodeStylex{##1}}}
 \renewcommand{\techterm}[1]{\textit{\CodeStylex{##1}}}
 \renewcommand{\term}[1]{\textit{##1}}
 \renewcommand{\grammarterm}[1]{\textit{##1}}
}

\lstnewenvironment{codeblocksmall}{\CodeBlockSetupSmall}{}



% An environment for command / program output that is not C++ code.
\lstnewenvironment{outputblock}{\lstset{language=}}{}

% A code block in which single-quotes are digit separators
% rather than character literals.
\lstnewenvironment{codeblockdigitsep}{
 \CodeBlockSetup
 \lstset{deletestring=[b]{'}}
}{}

% Permit use of '@' inside codeblock blocks (don't ask)
\makeatletter
\newcommand{\atsign}{@}
\makeatother

%%--------------------------------------------------
%% Indented text
\newenvironment{indented}[1][]
{\begin{indenthelper}[#1]\item\relax}
{\end{indenthelper}}

%%--------------------------------------------------
%% Library item descriptions
\lstnewenvironment{itemdecl}
{
 \lstset{escapechar=@,
 xleftmargin=0em,
 midpenalty=500,
 semicolonpenalty=-50,
 endpenalty=3000,
 aboveskip=2ex,
 belowskip=0ex	% leave this alone: it keeps these things out of the
				% footnote area
 }
}
{
}

\newenvironment{itemdescr}
{
 \begin{indented}[beginpenalty=3000, endpenalty=-300]}
{
 \end{indented}
}

%%--------------------------------------------------
%% add special hyphenation rules
\hyphenation{tem-plate ex-am-ple in-put-it-er-a-tor name-space name-spaces non-zero}

%%--------------------------------------------------
%% turn off all ligatures inside \texttt
\DisableLigatures{encoding = T1, family = tt*}



%%%%%%%%%%%%%%%%%%%% lundberg, normal-sized footnote text (we are not all young...)
\renewcommand{\footnotesize}{\normalsize}

%%%%%%%%%%%%%%%%%%%%%%%%%%%%%%%%%%
\newcommand{\dblquotes}[1]{``#1''}







\let\oldaddcontentsline\addcontentsline% Store \addcontentsline

\sloppy
\setcounter{tocdepth}{5}
%%%%%%%%%%%%%%%%%%%%%%%%%%%%%%%%%%%%%%%%
%\usepackage{showframe}
\usepackage{lipsum}
\usepackage{xcolor}

\usepackage{enumitem}
\newlist{myQuoteEnumerate}{enumerate}{2}% Set max nesting depth
\setlist[myQuoteEnumerate,1]{label=(\arabic*)}% Use numbers for level 1
\setlist[myQuoteEnumerate,2]{label=(\alph*)}%   Use letters for level 2

\newenvironment{MyQuote}{%
    \begin{myQuoteEnumerate}[resume=*,series=MyQuoteSeries]%
    \color{blue}%
    \item \begin{quote}%
}{%
    \end{quote}%
    \end{myQuoteEnumerate}%
}%
%%%%%%%%%%%%%%%%%%%%%%%%%%
%\iffalse
\usepackage{xcolor}
\usepackage{lineno}
\usepackage{lipsum}
\makeatletter
\renewcommand{\linenumberfont}{\fontsize{12pt}{3pt}\normalfont\smaller\color{gray}}
\makeatother
%\usepackage{lineno}
%\linenumbers
%\fi
%%%%%%%%%%%%%%%%%%%%%%%%
\linespread{1.2}
%%%%%%%%%%%%%%%%%%%%%%%%
\makeatletter
\renewcommand\tableofcontents{%
    \@starttoc{toc}%
}
\makeatother


\begin{document}
\vspace*{-9em}
\title{\tcode{Generalisation of nth_element to a range of nths}}
\author{
Johan Lundberg\\
}
\date{} %unused. Type date explicitly below.
\maketitle
\vspace*{-5em}

\begin{tabular}{ll}
Document \#:& P2375R0 \\
Date: &2021-05-14 \\
Project: & Programming Language C++ \\
Audience: & LEWG \\
Email: &\href{mailto:lundberj@gmail.com}{lundberj@gmail.com}
\end{tabular}

\iffalse %%% FOR ONLINE ABSTRACT
The paper proposes a generalisation of `std::nth_element`, taking a sorted range of iterators instead of a single nth element iterator, allowing arbitrary partial sorting of any sortable range.
The use and analysis of such algorithms is widespread and mature[Alsuwaiyel2001,Panh2002,lent1996,Shen1997]  and is available to Python programmers as numpy.partition[NpPart,NPImpl] since 2014.
\fi

\subsection*{Contents}
\tableofcontents

\section{Introduction}
The paper proposes a generalisation of \tcode{std::nth_element}, taking a sorted range of iterators instead of a single \tcode{nth} iterator, allowing arbitrary partial sorting of any sortable range.
The use and analysis of such algorithms is widespread and mature\cite{Alsuwaiyel2001,Panh2002,lent1996,Shen1997} and is available to Python programmers as numpy.partition\cite{NpPart,NPImpl} since 2014.

The single-nth \tcode{nth_element} algorithm has been part of the C++ standard library since the beginning\cite{StepLee95}, introduced as \dblquotes{\ldots  the  element  in  the  position  pointed  to  by nth  is  the  element  that  would  be  in  that position if the whole range were sorted. Also for any iterator i in the range [first, nth) and any iterator j in the range [nth, last) it holds that !(*i > *j) or comp(*i, *j) == false. It is linear on the average.}

A possible implementation of the range-of-nths algorithm is provided (section \ref{Implementation}).
It translates naturally to \tcode{std::ranges} versions just like \tcode{std::nth_element}, returning \tcode{last}.

To clarify what is new, the addition is called \tcode{std::nth_element\underline{s}} throughout the text, but is proposed to go as \tcode{nth_element}. 


\section{The algorithm}
\label{Implementation}
\label{Implement}

The proposal can be implemented with an algorithm that partitions the data on the midpoint of nths using the single-nth \tcode{nth_element} and proceeds to recurse into the two partitions. Ref \cite{Alsuwaiyel2001} considers this algorithm and concludes
 complexity \mbox{O(N log m)} where \mbox{N = last - first} and m is the number of unique elements in nths.
The same paper presents a parallel version, building on refs \cite{Akl1984,Akl1989,Shen1997}.

In many applications m is constant and the complexity as function of N alone is naturally linear on average. On the other hand, in the worst case m varies with N as \mbox{m = N}, and the whole container is sorted. For parallel versions (overloads taking an ExecutionPolicy) it is reasonable to leave freedom to implementers to do a full parallel sort and allow \mbox{O(N log N)}.

Python has numpy.partition\cite{NpPart} as their incarnation of \tcode{ntn_element}.
It supports a range-of-nths as proposed here and the implementation\cite{NPImpl} (in C++) uses \mbox{Introselect\cite{Musser1997}} by specification\footnote{numpy.partition and the in-place version numpy.ndarray.partion do not state complexity in terms of M (the size of nths) or m (the number of unique nths), but appears to be \mbox{$\sim$ N log M} for reasonable M, to become \mbox{$\sim$ N $\cdot$ M} for large M, such as M>1e4, N=1e6. }.

A possible implementation\footnote{
The implementation is not the point of the proposal but it may help explain it. Feedback would be very much appreciated. 
The algorithm described in \cite{Alsuwaiyel2001} is a little bit simpler since it takes \emph{unique} nths, something that is not reasonable to require here. The algorithms is \mbox{O(N log m)} except for a small N-independent term \mbox{O(M-m)} where M is the size of nths, due to the iterator comparisons and increments in \tcode{find_if_not} to skip over doublets in nths. The first if-statement is found in existing single-nth \tcode{nth_element} implementations, but here it also prevents some unnecessary bisections of nths.}
is shown below. Comparisons and projections can also be fed through in the most natural way. 

\begin{codeblock}
template< class RandomAccessIterator, class RandomAccessIterator2 >
constexpr void @\textbfx{nth_elements}@(RandomAccessIterator first, 
  RandomAccessIterator2 nth_first, RandomAccessIterator2 nth_last, 
  RandomAccessIterator last)
{
  if (last - first <= 32) { @\textbfx{std::sort(first, last)}@; return; }
  const auto nth_dist = nth_last - nth_first;
  if (nth_dist == 0 || *nth_first == last) return;
  const auto nth_mid = nth_first + nth_dist / 2;
  const auto at_nth_mid = *nth_mid;
  @\textbfx{nth_element(first, at_nth_mid, last)}@;
  @\textbfx{nth_elements(first, nth_first, nth_mid, at_nth_mid)}@;
  if (at_nth_mid != last){
    const auto nth_left = std::find_if_not(nth_mid, nth_last, 
      [at_nth_mid](auto v) {return v == at_nth_mid; });
    @\textbfx{nth_elements(at_nth_mid + 1, nth_left, nth_last, last)}@;
  }
}

\end{codeblock}


\section{Before/After}

Existing alternatives are to sort the whole container or to figure out a series of calls to e.g. \tcode{nth_element} and \tcode{partial_sort}. 
The examples below could be the linear time partitioning of messages to be
processed into fixed sized priority buckets, keeping or dropping remaining messages. Or finding the fastest 25, 100, and 1000 race participants in linear time. 
The partitions themselves form half open ranges so it's easy to e.g. sort and
print the 100th up to the 1000th fastest runners by name. 

Context: partitioning into a fixed number of slots
\begin{codeblock}
vector<decltype(v)::iterator> nths;
for(size_t slot=1; slot<16 ; ++slot){
	nths.push_back(v.begin()+ min(slot*2048,N));
}
\end{codeblock}
or at some other arbitrary iterators in the inclusive range [first,last].
\begin{codeblock}
auto nths=vector{v.begin()+25,v.begin()+100,v.begin()+1000}; 
\end{codeblock}

\subsection*{After \textnormal{ Simple and O(N)}}

\hspace{2ex}\framebox{\tcode{nth_elements(v, nths, pred);}}
\vspace*{1ex}

\subsection*{Before \textnormal{Alternative 1a: Hand-wired bisection for nths of known size. O(N) but messy}}
\begin{codeblock}
nth_element(v.begin(), nths[1], v.end(), pred);
nth_element(v.begin(), nths[0], nths[1], pred);
nth_element(nths[1]+1, nths[2], v.end(), pred);
\end{codeblock}
Did we get this right? Is it correct for repeated nths or empty v?

\subsection*{Before \textnormal{Alternative 1b: Hand-wired for size 3. O(N $\cdot$ M)}}
\begin{codeblock}
nth_element(v.begin(), nths[0],v.end(), pred);
nth_element(nths[0], nths[1], v.end(), pred);
nth_element(nths[1], nths[2], v.end(), pred);
\end{codeblock}
Did we get this right?

\subsection*{Before \textnormal{Alternative 2: Simple but O(N log N):}}
\begin{codeblock}
sort(v,pred);
\end{codeblock}

\newpage
\section{Applications}

It partitions into any number of partitions as shown in the previous section. Partitioning a bunch of ponies into several age groups, then sort one group by name.

\begin{codeblock}
stuct Pony{
  double littleness; 
  chrono::duration age;
  string name;
};
auto end=nth_elements(v, nths, std::greater{}, Pony::age);
std::sort(nths[3], nths[4], std::less{}, Pony::name);
\end{codeblock}

\subsubsection*{Partitioning with interpolation. Quantiles.}

% As such \tcode{nth_element(s)} are general algorithms on any sortable, and are to quantiles what
% \tcode{std::accumulate} and \tcode{std::inner_product} are to statistical mean and weighted mean.

\tcode{nth_elements} can be used to efficiently \emph{implement} the calculation of a single or a range of quantiles.
To do interpolation one may directly partition at two iterators at each quantile point. For example, partitioning N elements at a single quantile specified as a divisor d (where d=2 would be median and d=100 would mark the first percentile).
\begin{codeblock}
auto n = N == 0?0:N-1;
auto [q,r] = div(n, d);
auto nths=vector{first+q, first+q+(r>0)}; 
\end{codeblock}

\begin{codeblock}
auto last = nth_elements(v, nths, std::less{}, Pony::littleness);
if (nths[0]!=last){
  cout << nths[0]->name << " " << nths[1]->name ; 
  auto intrp_littleness = lerp(nths[0]->littleness, nths[1]->littleness, r*1.0/d);
}
\end{codeblock}

In the above we did floating point based interpolation, but 
one may stay in integer arithmetic\footnote{\tcode{i_lerp(auto a,auto b,auto r,auto d)\{return a+(r*(b-a))/d;\}}. Yes, there are other ways to express this depending on type, e.g. extra work to avoid overflow.}
 for example when working with chrono durations, iterators and indices. Any type the user knows how to interpolate.
 
\begin{codeblock}
auto last = nth_elements(v, nths, std::less{}, Pony::age);
if (nths[0]!=last){
  auto intrp_dur = i_lerp(nths[0]->age, nths[1]->age, r, d);
}
\end{codeblock}
\label{quantileanything}

In Python, \tcode{numpy.quantile}%
\footnote{
It defaults to the above division by N-1 to do linear interpolation but there's a plethora of variations (nine different modes supported by many statistical libraries and tools) on which indices to use, rounding and interpolation/\mbox{selection} and handling of edge cases. 
A good overview is found in P2119R0 commenting on the paper P1708R4 \dblquotes{Simple Statistical Functions} which proposes user-facing median and quantile similar to \tcode{numpy.partition}, returning by value (not via iterators).}
 takes a range of floating point quantile points in [0.0,1.0] and uses the previously mentioned multi-nth version of \tcode{numpy.partition}%
.



\newpage

\section{Wording and Synopsis }

\subsection{First wording draft \protect{\tcode{[alg.nth.element]}}}

\label{wording}

\textbf{1} Let comp be less{} and proj be identity{} for the overloads with no parameters by those names.

\textbf{2} Preconditions: [first, nth) and [nth, last) are valid ranges. For the overloads in namespace std, RandomAccessIterator meets the Cpp17ValueSwappable requirements ([swappable.requirements]), and the type of *first meets the Cpp17MoveConstructible (Table 30) and Cpp17MoveAssignable (Table 32) requirements.
\added{For the overloads taking a range [nths_first,nths_last), \newline RandomAccessIterator2 is a RandomAccess iterator, and *nths_first is convertible to \newline RandomAccessIterator. For every iterator i and j in the range [nths_first,nths_last), it holds that if (i<j) then !(*j<*i).}

\textbf{3} Preconditions: The elements e of [first, last) are partitioned with respect to the expression bool(invoke(comp, invoke(proj, e), value)).

\textbf{4} Effects: After nth_­element the element in the position pointed to by nth is the element that would be in that position if the whole range were sorted with respect to comp and proj, unless nth==last. Also for every iterator i in the range [first, nth) and every iterator j in the range [nth, last) it holds that: bool(invoke(comp, invoke(proj, *j), invoke(proj, *i))) is false.

\textbf{5} 
Complexity: For the overloads with no ExecutionPolicy, linear on average.
For the overloads with an ExecutionPolicy, O(N) applications of the predicate, and O(N log N) swaps, where N = last - first. 
\added{For overloads taking a range [nths_first,nths_last) but no ExecutionPolicy the complexity is approximately \mbox{O(N log m)} where m is the number of unique elements in [nths_first,nths_last).
For overloads taking a range [nths_first,nths_last) and an ExecutionPolicy the complexity is \mbox{O(N log N)}.}

\subsection{Synopsis -- \tcode{<algorithm> [algorithm.syn]}}

Added signatures:

\begin{codeblockAdd}
template<class RandomAccessIterator, class RandomAccessIterator2>
constexpr void nth_element(
RandomAccessIterator first, 
RandomAccessIterator2 nths_first, RandomAccessIterator2 nths_last,
RandomAccessIterator last);

template<class RandomAccessIterator, class RandomAccessIterator2, class Compare>
constexpr void nth_element(
RandomAccessIterator first, 
RandomAccessIterator2 nths_first, RandomAccessIterator2 nths_last,
RandomAccessIterator last, Compare comp);

template<class ExecutionPolicy, class RandomAccessIterator, class RandomAccessIterator2>
void nth_element(ExecutionPolicy&& exec,
RandomAccessIterator first, 
RandomAccessIterator2 nths_first, RandomAccessIterator2 nths_last,
RandomAccessIterator last);

template<class ExecutionPolicy, class RandomAccessIterator,
class RandomAccessIterator2, class Compare>
void nth_element(ExecutionPolicy&& exec,
RandomAccessIterator first, 
RandomAccessIterator2 nths_first, RandomAccessIterator2 nths_last,
RandomAccessIterator last, Compare comp);

namespace ranges {
  template<random_access_iterator I, sentinel_for<I> S, 
  random_access_range R2, class Comp = ranges::less, class Proj = identity>
  requires sortable<I, Comp, Proj>
  constexpr I nth_element(I first, R2&& nths, Comp comp = {}, Proj proj = {});

  template<random_access_range R, 
  random_access_range R2,
  class Comp = ranges::less, class Proj = identity>
  requires sortable<iterator_t<R>, Comp, Proj>
  constexpr safe_iterator_t<R>
  nth_element(R&& r, R2&& nths, Comp comp = {}, Proj proj = {});
}
\end{codeblockAdd}


\section{Questions and Answers}

Q: What's the best name? A: I suggest to reuse \tcode{nth_element} for discoverability but could as well be a separate name. Numpy calls both single nth and range-of-nths \dblquotes{\tcode{partition}}.

Q: What if \tcode{nths} or [first,last) is empty? A: \tcode{nth_elements} does nothing.

Q: What if some elements of \tcode{nths} are equal to last. A: As with \tcode{nth_element}, not a problem.

Q: What if some elements of \tcode{nths} are equal to each other A: By specification, not a problem.

\section*{Acknowledgements}
\addcontentsline{toc}{section}{Acknowledgements}

Many thanks to undisclosed proofreaders and to 
Albin Fredriksson and Marco Rubini for helpful discussions.


\renewcommand{\bibname}{References}
\bibliographystyle{abstract}
\bibliography{ref}

\renewcommand{\addcontentsline}[3]{}% Make \addcontentsline a no-op
\begin{thebibliography}{9}
\oldaddcontentsline{toc}{section}{References}

\bibitem[StepLee95]{StepLee95}
Alexander Stepanov and Meng Lee: The Standard Template Library.\\HP Laboratories Technical Report 95-11(R.1), November 14, 1995 \\
\url{http://stepanovpapers.com/STL/DOC.PDF }

\bibitem[Alsuwaiyel2001]{Alsuwaiyel2001} 
Muhammad H. Alsuwaiyel: An optimal parallel algorithm for the multiselection problem. 
Parallel Computing Volume 27, Issue 6, May 2001, Pages 861-865\\
\url{https://doi.org/10.1016/S0167-8191(00)00095-8 }
\bibitem[Akl1984]{Akl1984} 
S. G. Akl, Optimal parallel algorithms for computing convex hulls and for sorting, Computing, 33 (1984), 1-11.
\bibitem[Akl1989]{Akl1989} 
S. G. Akl, The Design and Analysis of Parallel Algorithms (PrenticeHall, Englewood Cliffs, New Jersey, 1989).
%M. L. Fredman and T. H. Spencer, Refined complexity analysis for heap operations, Journal of Computer and
%System Sciences, (1987), 269-284.
\bibitem[Shen1997]{Shen1997} 
H. Shen, Optimal parallel multiselection on EREW PRAM, Parallel Computing, 23(1997), 1987-1992.

\bibitem[NpPart]{NpPart}
Python numpy.partition \\
\url{https://numpy.org/doc/stable/reference/generated/numpy.partition.html }

\bibitem[NPImpl]{NPImpl}
The implementation of partition (multiple and single nth version) is found at
{\footnotesize \url{https://github.com/numpy/numpy/blob/v1.20.2/numpy/core/src/multiarray/item_selection.c#L1023 }}

\bibitem[Musser1997]{Musser1997}
David R. Musser, Introspective Sorting and Selection Algorithms\\
Software--Practice and Experience, (8): 983-993 (1997))\\
\url{https://www.cs.rpi.edu/~musser/gp/algorithms.html }

\bibitem[Panh2002]{Panh2002}
Alois Panholzer --
Analysis of multiple quickselect variants\\
Theoretical Computer Science
Volume 302, Issues 1–3, 13 June 2003, Pages 45-91\\
\url{https://doi.org/10.1016/S0304-3975(02)00729-6 }

\bibitem[lent1996]{lent1996}
Janice Lent, Hosam M.Mahmoud\\
Average-case analysis of multiple Quickselect: An algorithm for finding order statistics\\
Statistics \& Probability Letters \\
Volume 28, Issue 4, August 1996, Pages 299-310
\url{https://doi.org/10.1016/0167-7152(95)00139-5}

\end{thebibliography}
\let\addcontentsline\oldaddcontentsline% Restore \addcontentsline

\end{document}


